%%%%%%%%%%%%%%%%%%%%%%%%%%%%%%%%%%%%%%%%%
% Journal Article
% LaTeX Template
% Version 1.3 (9/9/13)
%
% This template has been downloaded from:
% http://www.LaTeXTemplates.com
%
% Original author:
% Frits Wenneker (http://www.howtotex.com)
%
% License:
% CC BY-NC-SA 3.0 (http://creativecommons.org/licenses/by-nc-sa/3.0/)
%
%%%%%%%%%%%%%%%%%%%%%%%%%%%%%%%%%%%%%%%%%

%----------------------------------------------------------------------------------------
%	PACKAGES AND OTHER DOCUMENT CONFIGURATIONS
%----------------------------------------------------------------------------------------

\documentclass[twoside]{article}

\usepackage{lipsum} % Package to generate dummy text throughout this template

\usepackage[sc]{mathpazo} % Use the Palatino font
\usepackage[T1]{fontenc} % Use 8-bit encoding that has 256 glyphs
\linespread{1.05} % Line spacing - Palatino needs more space between lines
\usepackage{microtype} % Slightly tweak font spacing for aesthetics

\usepackage[hmarginratio=1:1,top=32mm,columnsep=20pt]{geometry} % Document margins
\usepackage{multicol} % Used for the two-column layout of the document
\usepackage[hang, small,labelfont=bf,up,textfont=it,up]{caption} % Custom captions under/above floats in tables or figures
\usepackage{booktabs} % Horizontal rules in tables
\usepackage{float} % Required for tables and figures in the multi-column environment - they need to be placed in specific locations with the [H] (e.g. \begin{table}[H])
\usepackage{hyperref} % For hyperlinks in the PDF

\usepackage{lettrine} % The lettrine is the first enlarged letter at the beginning of the text
\usepackage{paralist} % Used for the compactitem environment which makes bullet points with less space between them

\usepackage{abstract} % Allows abstract customization
\renewcommand{\abstractnamefont}{\normalfont\bfseries} % Set the "Abstract" text to bold
\renewcommand{\abstracttextfont}{\normalfont\small\itshape} % Set the abstract itself to small italic text

\usepackage{titlesec} % Allows customization of titles
\renewcommand\thesection{\Roman{section}} % Roman numerals for the sections
\renewcommand\thesubsection{\Roman{subsection}} % Roman numerals for subsections
\titleformat{\section}[block]{\large\scshape\centering}{\thesection.}{1em}{} % Change the look of the section titles
\titleformat{\subsection}[block]{\large}{\thesubsection.}{1em}{} % Change the look of the section titles

\usepackage{fancyhdr} % Headers and footers
\pagestyle{fancy} % All pages have headers and footers
\fancyhead{} % Blank out the default header
\fancyfoot{} % Blank out the default footer
\fancyhead[C]{CS 189 $\bullet$ May 2014 $\bullet$ Final Project} % Custom header text
\fancyfoot[RO,LE]{\thepage} % Custom footer text

%----------------------------------------------------------------------------------------
%	TITLE SECTION
%----------------------------------------------------------------------------------------

\title{\vspace{-15mm}\fontsize{24pt}{10pt}\selectfont\textbf{Classifying the Disputed Federalist Papers}} % Article title

\author{
\large
\textsc{Larry Wu}\\%\thanks{A thank you or further information}\\[2mm] % Your name
\normalsize SID: 23598040 \\ % Your institution
%\normalsize \href{mailto:john@smith.com}{john@smith.com} % Your email address
\vspace{-5mm}
}
\date{}

%----------------------------------------------------------------------------------------

\begin{document}

\maketitle % Insert title

\thispagestyle{fancy} % All pages have headers and footers

%----------------------------------------------------------------------------------------
%	ABSTRACT
%----------------------------------------------------------------------------------------

\begin{abstract}

\noindent In this paper, we will use a bag of words model along with function words to train a K-Nearest Neighbors model to classify the 12 Disputed Federalist Papers.

\end{abstract}

%----------------------------------------------------------------------------------------
%	ARTICLE CONTENTS
%----------------------------------------------------------------------------------------

\begin{multicols}{2} % Two-column layout throughout the main article text

\section{Introduction}

\lettrine[nindent=0em,lines=3]{T}he Federalist papers are a series of 85 articles and essays published from October of 1787 to August 1788, which advocated the ratification of the United States Constitution. When the papers were initially published, the authors were a closely guarded secret. However, a list of the authors of each paper was eventually published in 1804 following Alexander Hamilton's death. The list revealed the three authors of the papers, Alexander Hamilton, James Madison, and John Jay, but there would eventually be some dispute over who wrote each of the essays. In 1818, Madison would publish his own list of the authors of each essay, which contradicted Hamilton's list for 12 of the essays.

There have been several previous attempts to classify the disputed essays. In 1964, Mosteller and Wallace used statistical inference by counting the frequency of 70 function words in the Federalist Papers, and found that Madison was the sole author of the 12 disputed essays. Robert A. Bosch and Jason A. Smith used a similar technique in 1998. They created a linear program and used cross-validation on every possible set of one, two, or three of the 70 function words to find the best separating hyperplane between Madison's confirmed essays and Hamilton's confirmed essays. They also concluded that all 12 of the disputed papers were written by Madison. Another paper by Glenn Fung, in 2003, used practically the same techniques, except with a linear Support Vector Machine to find a separating hyper plane, and also concluded that all 12 of the papers were written by Madison. Attempts to classify the Disputed Papers that do not rely on programming or statistical analysis are more divided in results. Some have also concluded that it was Madison who was the sole author of the 12 disputed papers, but others, like Joseph Rudman, have argued that the 12 disputed essays were actually a collaboration by Hamilton and Madison

%------------------------------------------------

\section{Methods}
In this paper we will be using K-Nearest Neighbors to classify the 12 Disputed Federalist Papers. In order to represent the papers in some kind of format a machine can process, we will be using a bag of words model, and rather than keeping track of the frequency of every word that appears in the Papers, we will only be keeping track of the frequency of 339 {\it Function Words}. Function words are words that have little lexical meaning, but serve to express grammatical relationships with other words within a sentence and can specify the mood of the writer. The list of 339 function words used is available from \href{http://www.sequencepublishing.com/academic.html}{www.sequencepublishing.com}. Included are lists of auxiliary verbs, conjunctions, determiners, prepositions, pronouns, and quantifiers.

All 85 federalist papers can be found online for free, and Project Gutenberg has a single text file contacting all 85 essays. The training set consisted of the 73 papers with confirmed authorship. The frequencies of each function word used by Madison and Hamilton were summed up, and then normalized. This resulted in two 'mean' sets of normalized function word frequencies, one for each author. We then computed this normalized function word frequency for each of the 12 disputed essay, and in order to classify the 12 essays, we computed the L2error between the disputed essay and the 'mean' sets for Madison and Hamilton. The classification that resulted in the least L2error was then selected as the classification for that essay.

%------------------------------------------------

\section{Results}

\begin{table}[H]
\caption{Function Words + Bag of Words + NN}
\centering
\begin{tabular}{llr}
\toprule
%\multicolumn{2}{c}{Classification Method} \\
\cmidrule(r){1-2}
Classification & Paper \# \\
\midrule
Madison &  $49$ \\
Madison & $50$ \\
Madison & $51$ \\
Madison & $52$ \\
Madison & $53$ \\
Madison & $54$ \\
Madison & $55$ \\
Madison & $56$ \\
Madison & $57$ \\
Madison & $58$ \\
Madison & $62$ \\
Madison & $63$ \\


\bottomrule
\end{tabular}
\end{table}







%------------------------------------------------

\section{Discussion}



\subsection{Function Words + Bag of Words + NN}

Like previous statistical and programming based attempts to classify the 12 disputed essays, we too have found that all 12 disputed essays seem to have been authored by Madison. To replicate these results for yourself, simply run 'python classifier.py', which will print out the classifications of the 12 disputed papers.

%----------------------------------------------------------------------------------------
%	REFERENCE LIST
%----------------------------------------------------------------------------------------

%\begin{thebibliography}{99} % Bibliography - this is intentionally simple in this template

%\bibitem[Figueredo and Wolf, 2009]{Figueredo:2009dg}
%Figueredo, A.~J. and Wolf, P. S.~A. (2009).
%\newblock Assortative pairing and life history strategy - a cross-cultural study.
%\newblock {\em Human Nature}, 20:317--330.
 
%\end{thebibliography}

%----------------------------------------------------------------------------------------

\end{multicols}

\end{document}